\documentclass{article}
\usepackage[utf8]{inputenc}
\usepackage{amsmath}
\title{Model Expl}
\author{Manu Kumar}
\date{February 2022}

\begin{document}

\section{Model Explanation}

The SIRDRe model is encapsulated by the following adjustment to the Kendrick-McCormack equations:

$$ \frac{dS}{dt} = - \beta SI + \gamma I $$
$$ \frac{dI}{dt} = \beta SI - \gamma I - \nu I + \mu R $$
$$ \frac{dR}{dt} = \gamma I - \mu dR $$
$$ \frac{dD}{dt} = \nu I $$
$$ \frac{dRe}{dt} = \mu R $$

The equilibrium point for this expression is clearly,

$$
\begin{bmatrix}
    S_0 \\ 
    I_0 \\
    R_0 \\
    D_0 \\
    Re_0
\end{bmatrix} 
= 
span \left\{ \begin{bmatrix} 1 \\ 0 \\ 0 \\ 0 \\ 0 \end{bmatrix} , \begin{bmatrix} 0 \\ 0 \\ 0 \\ 0 \\ 1 \end{bmatrix} \right\}
$$
Therfore, the initial population can be arbitrary in the analysis of equilbirium points. \\
Using the equilibrium point $S = s_0$, the Jacobian is:

$$J = \begin{bmatrix}
    0 & \gamma & 0 & 0 & 0 \\
    0 & \beta s_0 -\gamma - \nu & \mu & 0 & 0 \\
    0 & \gamma & -\mu & 0 & 0 \\
    0 & \nu & 0 & 0 & 0 \\
    0 & 0 & \mu & 0 & 0 
\end{bmatrix}$$

\begin{verbatim}
syms m n g b

% Jacobian of SIRRe model, b = beta * s_0
J(g, b, n, m) = [0  -b+g   0 0 0;
                 0 -b-g-n  m 0 0;
                 0    g   -m 0 0;
                 0    n    0 0 0;
                 0    0    m 0 0;];
[V, D] = eig(J)

\end{verbatim}
We obtain the following eigenvalues from this analysis:

$\lambda = \left\{ 0, 0, 0, -\frac{1}{2} \left( \beta s_0 + \gamma + \mu + \nu \pm \sqrt{D} \right) \right\} $ \\
where $D = \left( \beta s_0 + \gamma + \nu + \mu \right)^2 - 4\mu(\beta s_0 + \nu) $ \\
For most practical discussions, since $s_0$ describes a population, \\
$s_0 >> \beta, \gamma, \mu, \nu$ \\
Therefore, we can approximate our initial estimate of non-trivial values of $\lambda$ to:

$\lambda \approx -\frac{1}{2}\beta s_0 \pm \sqrt{\left( \beta s_0 \right)^2 - 4\mu \beta s_0}$ \\
Since the rate and population are both non-negative quantities, and non-zero for non-trivial cases,
the quantity $-\frac{1}{2}\beta s_0$ is negative. \\

\subsection{Real Eigenvalues}
There are three possible eigenvalue cases, determined by values of D.

$\Rightarrow D \geq 0$ 

$\Rightarrow s_0 \geq 4\mu$

\subsubsection{Both Negative}
We add the further constraint 

$\frac{1}{2} \beta s_0 > D$ 

$\Rightarrow \frac{1}{4} \left( \beta s_0 \right)^2 > \left( \beta s_0 \right)^2 - 4\mu \beta s_0$

$\Rightarrow \mu > \frac{3}{16} \beta s_0$ \\
But, as brought up earlier, $s_0$, in most practical cases, 
is much larger than $\beta$ and $\mu$. Thus, a factor of $\frac{3}{16}$ will not be enough to make $\mu$ ever outweigh the RHS.
Thus, two negative eigenvalues is impractical.

\subsubsection{One Zero, One Negative}
With the same thought process as the previous case, 
$\mu = \frac{3}{16} \beta s_0$ is also an impractical situation.

\subsubsection{One Positive, One Negative}
This is most likely, in the situation of real eigenvalues, that will arise.

\subsection{Complex Eigenvalues}
The contraint $s_0 < 4\mu$ cannot be practically achieved, 
hence the situation of complex eigenvalues will not arise. 

\end{document}
