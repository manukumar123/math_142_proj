\documentclass{article}
\usepackage[utf8]{inputenc}
\usepackage{amsmath}
\usepackage{amssymb}
\title{Model Expl}
\author{Manu Kumar}
\date{February 2022}

\begin{document}

\section{Model Explanation}

The SIRDRe model is encapsulated by the following adjustment to the Kendrick-McCormack equations:

$$ \frac{dS}{dt} = - \beta SI + \gamma I $$
$$ \frac{dI}{dt} = \beta SI - \gamma I - \nu I + \mu R $$
$$ \frac{dR}{dt} = \gamma I - \mu R $$
$$ \frac{dD}{dt} = \nu I $$
$$ \frac{dRe}{dt} = \mu R $$

Since we don't have access to the actual number of infected individuals, 
we have to assume the number of tested individuals is some weighted factor of the actual number of individuals:

$$ S(t) = \alpha(t)\tilde{S}(t) $$
$$ I(t) = \alpha(t)\tilde{I}(t) $$
$$ R(t) = \alpha(t)\tilde{R}(t) $$
$$ Re(t) = \alpha(t)\tilde{Re}(t) $$ 

Where $\alpha(t)$ is an 'optimism parameter'. 
We start at a realistic expectation of the proportion of the population getting tested 
(around $10^4$, akin to the bounds in the main paper), and plateaus at 1 over time, 
directed by the following equation:

$$ 1 - \left( 10^4 \left( \frac{tan^{-1}(x) + \pi / 2}{\pi} - 1\right) \right) $$

The final model we have access to:

$$ \frac{d\tilde{S}}{dt} = - \beta \tilde{S}(t)\tilde{I}(t) + \gamma \tilde{I}(t) $$
$$ \frac{d\tilde{I}}{dt} = \beta \tilde{S}(t)\tilde{I}(t) - \gamma \tilde{I}(t) - \nu \tilde{I}(t) + \mu \tilde{R}(t) $$
$$ \frac{d\tilde{R}}{dt} = \gamma \tilde{I}(t) - \mu \tilde{R}(t) $$
$$ \frac{dD}{dt} = \nu \alpha(t) \tilde{I}(t) $$
$$ \frac{d\tilde{Re}}{dt} = \mu \tilde{R}(t) $$

The equilibrium point for this expression is clearly,

$$
\begin{bmatrix}
    S_0 \\ 
    I_0 \\
    R_0 \\
    D_0 \\
    Re_0
\end{bmatrix} 
= 
span \left\{ \begin{bmatrix} 1 \\ 0 \\ 0 \\ 0 \\ 0 \end{bmatrix} \right\}
$$

$$
J = \begin{bmatrix}
    -\beta \tilde{I}(t) & -\beta \tilde{S}(t) + \gamma & 0 & 0 & 0 \\
    \beta \tilde{I}(t) & \beta \tilde{S}(t) -\gamma - \nu & \mu & 0 & 0 \\
    0 & \gamma & -\mu & 0 & 0 \\
    0 & \nu & 0 & 0 & 0 \\
    0 & 0 & \mu & 0 & 0 
\end{bmatrix}
$$

Therfore, the initial population can be arbitrary in the analysis of equilbirium points. \\
Using the equilibrium point $S = s_0$, the Jacobian is:

$$J_{eq} = \begin{bmatrix}
    0 & -\beta s_0 + \gamma & 0 & 0 & 0 \\
    0 & \beta s_0 -\gamma - \nu & \mu & 0 & 0 \\
    0 & \gamma & -\mu & 0 & 0 \\
    0 & \nu & 0 & 0 & 0 \\
    0 & 0 & \mu & 0 & 0 
\end{bmatrix}$$

\begin{verbatim}
syms m n g b

% Jacobian of SIRRe model, b = beta * s_0
J(g, b, n, m) = [0  -b+g   0 0 0;
                 0  b-g-n  m 0 0;
                 0    g   -m 0 0;
                 0    n    0 0 0;
                 0    0    m 0 0;];
[V, D] = eig(J)

\end{verbatim}
We obtain the following eigenvalues from this analysis:

$\lambda = \left\{ 0, 0, 0, -\frac{1}{2} \left( \beta s_0 + \gamma + \mu + \nu \pm \alpha(t)\sqrt{D} \right) \right\} $ \\
where $D = \left( \beta s_0 + \gamma + \nu + \mu \right)^2 + 4\mu(\beta s_0 + \nu) $ \\
For most practical discussions, since $s_0$ describes a population, \\
$s_0 >> \beta, \gamma, \mu, \nu$ \\
Therefore, we can approximate our initial estimate of non-trivial values of $\lambda$ to:

$\lambda \approx \frac{1}{2} \left(-\beta s_0 \pm \alpha(t)\sqrt{\left( \beta s_0 \right)^2 + 4\mu \beta s_0} \right)$ \\
Since the rate and population are both non-negative quantities, and non-zero for non-trivial cases,
the quantity $-\beta s_0$ is negative.

\subsection{Eigenvalues}
Since all the rates and poplations are non-negative quantities, the quantity $D > 0$.
For estimates of $\alpha(t)$ earlier in the spread of the pandemic, 
our model assumes it is quite large, and plateaus to 1. Hence, it is clear that at any t,
$\alpha(t) \sqrt{\left( \beta s_0 \right)^2 + 4\mu \beta s_0} > \beta s_0$. \\
$\therefore$ we would assume the model is \textbf{asymptotically unstable}, 
and in the long run the epidemic will have a minimal effect on the population.

\end{document}
